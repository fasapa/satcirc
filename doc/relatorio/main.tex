% Exemplo de relatório técnico e outros documentos do INF-UFG 
% com texto em português formatado com LaTeX

\documentclass[a4paper, 12pt]{article}

% MODIFICAÇÕES CLASSE INF
\usepackage[english,portuguese,brazil]{babel}
\usepackage[T1]{fontenc}
\usepackage[utf8]{inputenc}
\usepackage{graphicx}
\usepackage{longtable}
\usepackage{float}
\usepackage{fancyvrb}
\usepackage{hyperref}
\usepackage{url} 
\usepackage[vlined,titlenumbered,algo2e,ruled,portuguese]{algorithm2e} 
\usepackage{indentfirst}
\usepackage{amsmath}
\usepackage{listings}
\lstset{
  basicstyle=\ttfamily\footnotesize,
  mathescape
}
\usepackage{newtxtext}
\usepackage{newtxmath}
\usepackage{times}


% MARGENS
\setlength{\voffset}{-1in}
\setlength{\hoffset}{-1in}

\setlength{\oddsidemargin}{3.0cm}
\setlength{\evensidemargin}{3.0cm}
\setlength{\textwidth}{\paperwidth}
\addtolength{\textwidth}{-\oddsidemargin}
\addtolength{\textwidth}{-2.0cm}

\setlength{\topmargin}{0.0cm}
\setlength{\headheight}{1.5cm}
\setlength{\headsep}{1.5cm}

\setlength{\textheight}{\paperheight}
\addtolength{\textheight}{-\topmargin}
\addtolength{\textheight}{-\headheight}
\addtolength{\textheight}{-\headsep}
\addtolength{\textheight}{-2.0cm}

\setlength{\parindent}{1cm}
% MARGENS

% AMBIENTES
\renewcommand{\theequation}{\thechapter-\arabic{equation}}
\newtheorem{definition}{Defini\c{c}\~{a}o}[section]
\newtheorem{theorem}[definition]{Teorema}
\newtheorem{corollary}[definition]{Corol\'{a}rio}
\newtheorem{lemma}[definition]{Lema}
\newtheorem{proposition}[definition]{Proposi\c{c}\~{a}o}
\newtheorem{comment}[definition]{Observac{c}\~{a}o}
\newtheorem{conjecture}[definition]{Conjectura}
\newenvironment{proof}{\par \noindent \textit{Prova}.}{\hfill $\Box$\newline}
% % AMBIENTES

\floatstyle{ruled}
\newfloat{codigo}{tbp}{lop}
\floatname{codigo}{Código}

% \renewenvironment{abstract}{%
% %\selectlanguage{english}
%       \list{}{%\advance\topsep by0.35cm\relax%\small
%       \leftmargin=0.8cm
%       \labelwidth=\z@
%       \listparindent=\z@
%       \itemindent\listparindent
%       \rightmargin\leftmargin}\item[\hskip\labelsep
%                                     \bfseries\itshape Abstract.]\itshape}%
%       {\endlist}

% \newenvironment{keywords}{%
%       \list{}{\advance\topsep by0.35cm\relax%\small
%       \leftmargin=0.8cm
%       \labelwidth=\z@
%       \listparindent=\z@
%       \itemindent\listparindent
%       \rightmargin\leftmargin}\item[\hskip\labelsep
%                                     \bfseries Keywords:]}%
%       {\endlist}

\makeatletter
\newenvironment{resumo}{%
  \list{}{%\advance\topsep by0.35cm\relax%\small
    \leftmargin=0.8cm
    \labelwidth=\z@
    \listparindent=\z@
    \itemindent\listparindent
    \rightmargin\leftmargin}\item[\hskip\labelsep
  \bfseries\itshape Resumo.]\itshape}%
{\endlist}


\newenvironment{palavras-chave}{%
  \list{}{%\advance\topsep by0.35cm\relax%\small
    \leftmargin=0.8cm
    \labelwidth=\z@
    \listparindent=\z@
    \itemindent\listparindent
    \rightmargin\leftmargin}\item[\hskip\labelsep
  \bfseries Palavras-Chave:]}%
{\endlist}
\makeatother

\begin{document}

%%---------------------------------------------------------- TITULO E AUTOR DA PAG 1 %
% TITULO e NOMES DOS AUTORES, completos, para a pagina 1.
% Use "\\" para quebrar linhas, "\and" para separar autores.
%
\title{\bf Redação de Relatórios Técnicos \\e Outros Documentos do
  \\ Instituto de Informática da UFG Usando \LaTeX}

\author{Fabrício \thanks{fabricio@email.com} \and
  Felipe \thanks{felipe@email.com} \and
  Arthur \thanks{arthur@emaiil.com}}
\date{}

\maketitle
%---------------------------------------------------------- TITULO E AUTOR DA PAG 1 %


%---------------------------------------------------------- CABECALHO DAS PAGINAS %
% Nomes de autores ABREVIADOS e título ABREVIADO, para cabecalhos em cada página.
% \markboth{\small de Carvalho e Longo}{\small Relatórios Técnicos Usando \LaTeX}
% \pagestyle{myheadings}
%---------------------------------------------------------- CABECALHO DAS PAGINAS %


%----------------------------------------------------------ABSTRACT %
% \begin{abstract}
% This meta-paper describes how to use the class that defines the model for Technical Reports and other documents, according to the patterns adopted by the Instituto de Informática of UFG. An abstract and "resumo" should be added. In both cases, abstract (and ``resumo'') should not have more than 10 lines and must be in the first page of the paper. 
% \end{abstract}

% \begin{keywords}
% Technical Report, \LaTeX.
% \end{keywords}
%----------------------------------------------------------ABSTRACT %

%----------------------------------------------------------RESUMO %
\begin{resumo} 
Este meta-artigo descreve como usar a classe que define o modelo para a confecção de Relatórios Técnicos e outros documentos, segundo o padrão adotado pelo Instituto de Informática da UFG. É solicitada a escrita de resumo e \textit{abstract}. O autor deve tomar cuidado para que o resumo e o \textit{abstract} não ultrapassem 10 linhas cada, sendo que ambos devem estar na primeira página do relatório.
\end{resumo}

\begin{palavras-chave}
Relatório Técnico, \LaTeX.
\end{palavras-chave}
% ----------------------------------------------------------RESUMO

\section{Introdução}
\label{sec:introducao}

Na atualidade os sistemas digitais estão crescendo exponencialmente em complexidade para reduzir o tamanho dos componentes. Um problema que isso gera é a garantia do funcionamento de todas as partes e do conjunto delas. Nesse contexto podem-se usar métodos de inteligência artificial (IA) para deixar essa resolução mais eficiênte.

Porêm apenas a aplicação direta de IA não é suficiente, precisa-se modelar o problema e processar as informações da tal forma que o algoritmo possa chegar a uma conclusão de forma eficiênte. Para tal uma alternativa bastante utilizada é o formato de satisfabilidade booleana (SAT) que interage bem com uma busca informada. 

O SAT (problema de satisfatibilidade booleana) é um problema clássico de decisão da lógica proposicional, NP completo, que diz respeito ao teste de uma expressão booleana que só aceita AND (e lógico), OR (ou lógico) e NOT (negação), variáveis e parênteses modelados no formado de conjunção de disjunções. Consiste em determinar se existe algum estado das variáveis em que a saída da expressão é positiva (satistativel).

Além do modelo de satisfabilidade também é necessário um formato de interpretação do circuito envolvido no problema, por isso foi elaborado um modelo básico de hardware baseado em \textit{Hardware Description Language} (HDL). O formato de descrição de circuito usado recebe um circuito na forma (porta NAND como exemplo):
\begin{lstlisting}
  Nome_Circuito [intput;output] {componentes_do_circuito}
  NAND [a b; c] {
    a b AND d;
    d NOT c;
  }
\end{lstlisting}

\section{SAT}
\label{sec:sat}

\subsection{Problema de satisfatibilidade booliana}\label{sec:ssat}

\subsection{MiniSat}
\label{sec:minisat}

Minisat é um minimalistico, open-source resolvedor de problemas de satisfatibilidade booleana (SAT), desenvolvido para 
ajudar pesquisadores e desenvolvedores a começarem desenvolvimento em SAT's. Seu funcionamento é basiado em backtracking
por conflito de analises e aprendisado.

Os componentes do resolvedor do MiniSat podem ser divididos conceptualmente em três categorias. A primeira é a representação,
de alguma forma a instância do SAT deve ser representada por estruturas de dados. A segunda é a inferência, o resolvedor necessita
de mecanismos para computar e propagar as implicações do estado atual de informações. A terceira é pesquisa, geralmente combinado
com inferência, pesquisa é necessária para encontrar informação.

Para realizar pesquisas, assunções são feitas, atribuindo valores a variáveis selecionadas até que a propagação detecte um conflito.
Nesse ponto uma cláusula conflito é construída e adicionada ao problema SAT.

O MiniSat utiliza como base o algoritmo de pesquisa de Davis-Putman-Longemann-Loveland (DPLL), baseado em 
retrocesso (backtracking), que serve para decidir a satisfatibilidade de formulas da logica proposicional em formula normal
conjutiva, ou seja, para resolver problemas SAT.

O algoritimo de backtracking escolhe um literal $\Phi$ e lhe da o valor de \texttt{TRUE}, simplificando a formula e depois
recursivamente checando se a formula simplificada é satisfatível. Caso seja, a formula original também é
satisfatível, caso contrario, a mesma checagem recursiva é feita mas dando o valor \texttt{FALSE} dado ao literal $\phi$.
Assim dividindo o problema em dois, simplificados, sub-problemas.

Se uma clausula contem apenas um literal, então essa clausula so pode ser satisfeita se for associado o valor
\texttt{TRUE} ao literal. Quando uma variável proposicional ocorre com apenas uma polaridade, ela é chamada de pura. Variáveis puras
podem ser associadas de um jeito onde toda clausula que as contem resultam em \texttt{TRUE}, logo tais clausulas podem
ser deletadas da busca.
\begin{lstlisting}
  funcao DPLL($\Phi$, u)
   se todas as clausulas de Y forem verdadeiras 
       entao retorne TRUE;
       
   se alguma clausula de Y for falsa
       entao retorne FALSE;
       
   se ocorrer uma clausula unitaria c em Y
       entao retorne DPLL(atribuicao(c,Y), (u and c));
       
   se ocorrer um literal puro c em Y
       entao retorne DPLL(atribuicao(c,Y), (u and c));
       
   c := escolha_literal(Y);
   
   retorne DPLL(atribuicao(c,Y), (u and c)) ou
           DPLL(atribuicao(-c,Y), (u and -c));
\end{lstlisting}

 No pseudocódigo acima, \texttt{atribuição(c, $\Phi$)} é uma função que retorna uma fórmula obtida pela substituição de cada ocorrência
 de \texttt{c} por \texttt{TRUE}, e cada ocorrência do literal oposto por falso na fórmula \texttt{Y}, e em seguida, simplificando a fórmula resultante.
 A função DPLL do pseudocódigo retorna verdadeiro se a atribuição final satisfaz a fórmula ou falso se tal atribuição não satisfaz 
 a fórmula. Em uma implementação real, a atribuição satisfatível também é retornada no caso de sucesso (esta foi omitida para maior clareza).

\section{Implementação}
\label{sec:implementacao}


\subsection{Compilador}
\label{sec:Compilador}
O sistema de leitura do circuito e da Forma Normal Conjuntiva (CNF) foram implementados utilizando as ferramentas flex\footnote{https://github.com/westes/flex} e bison\footnote{https://www.gnu.org/software/bison/}.% , os frameworks são utilizados em conjunto com muita frequência, então existem funções internas neles que facilitam a interação entre os dois.

O flex é um analisador lexico% da GNU (fab: NÃO É)
que analisa o texto e o separa em tokens baseados em expressões regulares (formados que descrevem um conjunto de letras e números em quantidade e ordem específicada). O bison é um parser generator, ou analisador sintático, que recebe esses tokens do analisador lexico, e verifica se estão em concordância com a gramática
especificada.
% determina o significado dos conjuntos de tokens e os converte para um formato utilizavel pelo sistema SAT.

Para o circuito ser analisado ele foi separado em tokens de variaveis representando os fios de entrada e saida e identificadores representando o nome dos circuitos envolvidos, já que um circuito maior pode ser representado por um conjunto de circuitos menores. Abaixo está a gramática utilizada no formato
Backus-Naur.
\begin{lstlisting}
  circuito    ::=
     ID '[' variaveis ';' variaveis ']' '{' componentes '}'
  componentes ::= componente | componentes componente
  componente  ::= variaveis ID variaveis ';'
  variaveis   ::= VAR | variaveis VAR
\end{lstlisting}

No caso do CNF o analisador léxico separa basicamento o nome que representa o circuito representado e os números do conjunto, e o papel do analisador sintático é de separar os números em suas respectivas cláusulas e converte-las para o formato implementado no sistema SAT. Abaixo está a gramática utilizada.
\begin{lstlisting}
  cnf       ::= ID NUM clausulas
  clausulas ::= clausula | clausulas clausula
  clausula  ::= '(' variaveis ')'
  variaveis ::= NUM | variaveis NUM
\end{lstlisting}


\section{Resultados}
\label{sec:resultados}

No programa já estão incluidas as portas básicas AND, OR e NOT. Circuitos mais
complexos devem ser construidos a partir de tais portas, montando uma
biblioteca de circuitos disponíveis para uso posterior.

\subsection{Compilação}
\label{sec:compilacao}

O processo de compilação é relativamente simples. O programa aplica as
transformações de Tseytin para portas já definidas, emitindo, no final
a CNF do circuito descrito na entrada. Para ilustrar os resultados
iremos utilizar a porta NAND como exemplo.
\begin{lstlisting}
  cat nand.txt
  NAND [a b; c] {
    a b AND d;
    d NOT c;
  }
\end{lstlisting}
A estrutura do input é formada pelo nome do circuito (deve começar com uma
letra maiúscula), seguido da descrição das entradas e saidas (aqui a b são
portas de entrada enquanto c saida) e sucessivamente uma lista de componentes
do circuito (portas AND e NOT).

Para a obtenção da CNF executa-se o comando de compilação e o resultado é
emitido para saida padrão do sistema.
\begin{lstlisting}
  satcirc c nand.txt
  NAND 3 (-1 -2 4)(1 -4)(2 -4)(-4 -3)(4 3)
\end{lstlisting}
Como visto na descrição do NAND acima o usuário pode introduzir variáveis
para descrever saidas de subcircuitos para posterior utilização interna.

\subsection{Especificação}
\label{sec:espec}

A verificação de propriedades é feita utilizando um arquivo auxiliar. Como
exemplo, queremos verificar se nossa especficação de porta NAND está correta.
Vamos analisar em quais condições a saida é verdade(1):
\begin{lstlisting}
  cat nand_ver.txt
  c
  satcirc v nand.txt nand_ver.txt
  SAT a(0) b(0) c(1)
\end{lstlisting}
Para saida c(1) temos que é satisfatível (SAT) para as entradas a(0) e b(0),
como de acordo com a especificação da porta NAND. Entretanto está não é a
única possibilidade de c(1), para encontrar as outras possibilidades devemos
introduzir o resultado negado no arquivo de verificação:
\begin{lstlisting}
  cat nand_ver.txt
  c
  a b -c
  satcirc v nand.txt nand_ver.txt
  SAT a(0) b(1) c(1)
\end{lstlisting}
Para cada resultado, uma nova linha negada é introduziada no arquivo de
verificação:
\begin{lstlisting}
  cat nand_ver.txt
  c
  a b -c
  a -b -c
  satcirc v nand.txt nand_ver.txt
  SAT a(1) b(0) c(1)
\end{lstlisting}
e finalmente:
\begin{lstlisting}
  cat nand_ver.txt
  c
  a b -c
  a -b -c
  -a b -c
  satcirc v nand.txt nand_ver.txt
  UNSAT
\end{lstlisting}
Temos que este problema não é satisfatível (UNSAT), ou seja, não existem mais
possibilidades da saida c(1) de acordo com as especificações de
verificação. Abaixo temos o análogo para saida c(0).
\begin{lstlisting}
  cat nand_ver.txt
  -c
  satcirc v nand.txt nand_ver.txt
  SAT a(1) b(1) c(0)
  
  cat nand_ver.txt
  -c
  -a -b c
  satcirc v nand.txt nand_ver.txt
  UNSAT
\end{lstlisting}

% \bibliography{exemplo}
\end{document}
