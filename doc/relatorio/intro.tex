\section{Introdução}
\label{sec:introducao}

O SAT (problema de satisfatibilidade booleana) é um problema clássico de descisão da lógica proposicional, NP completo, que diz respeito ao teste de uma expressão booleana que só aceita AND (e lógico), OR (ou lógico) e NOT (negação), variáveis e parênteses modelados no formado de conjunção de disjunções.

O objetivo deste problema é testar se existe algum estado das variáveis em que a saída da expressão é positiva (satistativel).


