\section{Implementação}
\label{sec:implementacao}


\subsection{Compilador}
\label{sec:Compilador}
O sistema de leitura do circuito e da Forma Normal Conjuntiva (CNF) foram implementados utilizando as ferramentas flex\footnote{https://github.com/westes/flex} e bison\footnote{https://www.gnu.org/software/bison/}.% , os frameworks são utilizados em conjunto com muita frequência, então existem funções internas neles que facilitam a interação entre os dois.

O flex é um analisador lexico% da GNU (fab: NÃO É)
que analisa o texto e o separa em tokens baseados em expressões regulares (formados que descrevem um conjunto de letras e números em quantidade e ordem específicada). O bison é um parser generator, ou analisador sintático, que recebe esses tokens do analisador lexico, e verifica se estão em concordância com a gramática
especificada.
% determina o significado dos conjuntos de tokens e os converte para um formato utilizavel pelo sistema SAT.

Para o circuito ser analisado ele foi separado em tokens de variaveis representando os fios de entrada e saida e identificadores representando o nome dos circuitos envolvidos, já que um circuito maior pode ser representado por um conjunto de circuitos menores. Abaixo está a gramática utilizada no formato
Backus-Naur.
\begin{lstlisting}
  circuito    ::=
     ID '[' variaveis ';' variaveis ']' '{' componentes '}'
  componentes ::= componente | componentes componente
  componente  ::= variaveis ID variaveis ';'
  variaveis   ::= VAR | variaveis VAR
\end{lstlisting}

No caso do CNF o analisador léxico separa basicamento o nome que representa o circuito representado e os números do conjunto, e o papel do analisador sintático é de separar os números em suas respectivas cláusulas e converte-las para o formato implementado no sistema SAT. Abaixo está a gramática utilizada.
\begin{lstlisting}
  cnf       ::= ID NUM clausulas
  clausulas ::= clausula | clausulas clausula
  clausula  ::= '(' variaveis ')'
  variaveis ::= NUM | variaveis NUM
\end{lstlisting}
