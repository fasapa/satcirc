\section{Implementação}
\label{sec:implementacao}

\subsection{Compilador}
\label{sec:Compilador}
O sistema de leitura do circuito e da Forma Normal Conjuntiva (CNF) foram implementados utilizando as ferramentas flex\footnote{https://github.com/westes/flex} e bison\footnote{https://www.gnu.org/software/bison/}.% , os frameworks são utilizados em conjunto com muita frequência, então existem funções internas neles que facilitam a interação entre os dois.

O flex é um analisador lexico% da GNU (fab: NÃO É)
que analisa o texto e o separa em tokens baseados em expressões regulares (formados que descrevem um conjunto de letras e números em quantidade e ordem específicada). O bison é um parser generator, ou analisador sintático, que recebe esses tokens do analisador lexico, e verifica se estão em concordância com a gramática
especificada.
% determina o significado dos conjuntos de tokens e os converte para um formato utilizavel pelo sistema SAT.

Para o circuito ser analisado ele foi separado em tokens de variaveis representando os fios de entrada e saida e identificadores representando o nome dos circuitos envolvidos, já que um circuito maior pode ser representado por um conjunto de circuitos menores. Abaixo está a gramática utilizada no formato
Backus-Naur.
\begin{lstlisting}
  circuito    ::=
     ID '[' variaveis ';' variaveis ']' '{' componentes '}'
  componentes ::= componente | componentes componente
  componente  ::= variaveis ID variaveis ';'
  variaveis   ::= VAR | variaveis VAR
\end{lstlisting}

No caso do CNF o analisador léxico separa basicamento o nome que representa o circuito representado e os números do conjunto, e o papel do analisador sintático é de separar os números em suas respectivas cláusulas e converte-las para o formato implementado no sistema SAT. Abaixo está a gramática utilizada.
\begin{lstlisting}
  cnf       ::= ID NUM clausulas
  clausulas ::= clausula | clausulas clausula
  clausula  ::= '(' variaveis ')'
  variaveis ::= NUM | variaveis NUM
\end{lstlisting}

\subsection{Transformação de Tseytin}\label{sec:tseytin}
Todo circuito digital combinatório pode ser escrito como uma formula da lógica proposicional, utilizando
os conectivos usuais. Entretanto SAT espera uma CNF como entrada, ou seja, contendo apenas
os operadores binários $\land$ e $\lor$. A transformação é realizada utilizando equivalências
lógicas, como as leis de De Morgan, até que se obtenha o formato esperado. Um ponto negativo deste
método é a explosão de termos causado pela distributividade da disjunção na conjunção:
$P \lor (Q \land R) \equiv (P \lor Q) \land (P \lor R)$, gerando um aumento exponêncial no tamanho
do termo.

Para contornar este problema utilizamos uma transformação proposta por Tseytin
\footnote{G.S. Tseytin: On the complexity of derivation in propositional calculus. In: Slisenko, A.O. (ed.) Studies in Constructive Mathematics and Mathematical Logic, Part II, Seminars in Mathematics, pp. 115–125. Steklov Mathematical Institute (1970).}. Neste método o tamanho da formula final é linear
com o tamanho do circuito. A transformação não gera formulas equivalentes, pois há introdução
de variáveis novas, mas mantem a satisfatibilidade, ou seja, a formula final é satistatível se, e
somente se, a formula inicial for satisfatível. A transformação consiste em três partes
(exemplo abaixo): idêntificação de todas subformulas, introduzir uma nova variável para cada
subformula e conjunção de todas subformulas. Formula original,
\begin{equation*}
  \phi := ((p \lor q) \land r) \leftarrow (\neg s)
\end{equation*}
idêntificação das subformulas,
\begin{eqnarray*}
  &\neg s \\
  &p \lor q \\
  &(p \lor q) \land r \\
  &((p \lor q) \land r) \leftarrow (\neg s)
\end{eqnarray*}
introdução de variáveis novas,
\begin{eqnarray*}
  x_1 \leftrightarrow& \neg s \\
  x_2 \leftrightarrow& p \lor q \\
  x_3 \leftrightarrow& (p \lor q) \land r \\
  x_4 \leftrightarrow& ((p \lor q) \land r) \leftarrow (\neg s)  
\end{eqnarray*}
conjunção de todas as formulas e substituição em $\phi$,
\begin{eqnarray*}
  T(\phi) := x_4 \land
  (x_4 \leftrightarrow x_3 \leftarrow x_1) \land
  (x_3 \leftrightarrow x_2 \land r ) \land
  (x_2 \leftrightarrow p \lor q) \land
  (x_1 \leftrightarrow \neg s)
\end{eqnarray*}
deste ponto em diante utiliza-se as transformações usuais da lógica proposicional.