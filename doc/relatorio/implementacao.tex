\section{Implementação}
\label{sec:implementacao}


\subsection{Compilador}
\label{sec:Compilador}


			O sistema do Circuito e da Forma Normal Conjuntiva foram interpretados por meio de compiladores de texto implementados especificamente para o SAT, para o desenvolvimento desses interpretadores foram usadas as ferramentas Flex e Bison, os frameworks são utilizados em conjunto com muita frequência, então existem funções internas neles que facilitam a interação entre os dois.
			
		O Flex é analisador lexico da GNU que analisa o texto e o separa em tokens baseados em expressões regulares (formados que descrevem um conjunto de letras e números em quantidade e ordem específicada ).
		
			O Bison é um parser ou analisador sintático que recebe esses tokens do analisador lexico, determina o significado dos conjuntos de tokens e os converte para um formato utilizavel pelo sistema SAT.
	
			Para o Circuito ser analisado ele foi separado em tokens de variaveis representando os fios de entrada e saida e identificadores representando o nome dos circuitos envolvidos , já que um circuito maior pode ser representado por um conjunto de circuitos menores.

			No caso da Forma Normal Conjuntiva (Cnf) o analisador léxico separa basicamento o nome que representa o circuito representado e os números do conjunto , e o papel do analisador sintático é de separar os números em suas respectivas cláusulas e converte-las para o formato implementado no sistema SAT.

