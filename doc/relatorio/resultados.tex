\section{Resultados}
\label{sec:resultados}

No programa já estão incluidas as portas básicas AND, OR e NOT. Circuitos mais
complexos devem ser construidos a partir de tais portas, montando uma
biblioteca de circuitos disponíveis para uso posterior.

\subsection{Compilação}
\label{sec:compilacao}

O processo de compilação é relativamente simples. O programa aplica as
transformações de Tseytin para portas já definidas, emitindo, no final
a CNF do circuito descrito na entrada. Para ilustrar os resultados
iremos utilizar a porta NAND como exemplo.
\begin{lstlisting}
  cat nand.txt
  NAND [a b; c] {
    a b AND d;
    d NOT c;
  }
\end{lstlisting}
A estrutura do input é formada pelo nome do circuito (deve começar com uma
letra maiúscula), seguido da descrição das entradas e saidas (aqui a b são
portas de entrada enquanto c saida) e sucessivamente uma lista de componentes
do circuito (portas AND e NOT).

Para a obtenção da CNF executa-se o comando de compilação e o resultado é
emitido para saida padrão do sistema.
\begin{lstlisting}
  satcirc c nand.txt
  NAND 3 (-1 -2 4)(1 -4)(2 -4)(-4 -3)(4 3)
\end{lstlisting}
Como visto na descrição do NAND acima o usuário pode introduzir variáveis
para descrever saidas de subcircuitos para posterior utilização interna.

\subsection{Especificação}
\label{sec:espec}

A verificação de propriedades é feita utilizando um arquivo auxiliar. Como
exemplo, queremos verificar se nossa especficação de porta NAND está correta.
Vamos analisar em quais condições a saida é verdade(1):
\begin{lstlisting}
  cat nand_ver.txt
  c
  satcirc v nand.txt nand_ver.txt
  SAT a(0) b(0) c(1)
\end{lstlisting}
Para saida c(1) temos que é satisfatível (SAT) para as entradas a(0) e b(0),
como de acordo com a especificação da porta NAND. Entretanto está não é a
única possibilidade de c(1), para encontrar as outras possibilidades devemos
introduzir o resultado negado no arquivo de verificação:
\begin{lstlisting}
  cat nand_ver.txt
  c
  a b -c
  satcirc v nand.txt nand_ver.txt
  SAT a(0) b(1) c(1)
\end{lstlisting}
Para cada resultado, uma nova linha negada é introduziada no arquivo de
verificação:
\begin{lstlisting}
  cat nand_ver.txt
  c
  a b -c
  a -b -c
  satcirc v nand.txt nand_ver.txt
  SAT a(1) b(0) c(1)
\end{lstlisting}
e finalmente:
\begin{lstlisting}
  cat nand_ver.txt
  c
  a b -c
  a -b -c
  -a b -c
  satcirc v nand.txt nand_ver.txt
  UNSAT
\end{lstlisting}
Temos que este problema não é satisfatível (UNSAT), ou seja, não existem mais
possibilidades da saida c(1) de acordo com as especificações de
verificação. Abaixo temos o análogo para saida c(0).
\begin{lstlisting}
  cat nand_ver.txt
  -c
  satcirc v nand.txt nand_ver.txt
  SAT a(1) b(1) c(0)
  
  cat nand_ver.txt
  -c
  -a -b c
  satcirc v nand.txt nand_ver.txt
  UNSAT
\end{lstlisting}

\section{Considerações finais}
Em casos reais de verificação de circuitos, a quantidade
de variáveis chegaria a casa de milhares ou centenas de milhares,
assim, SAT solvers são exemplos de que, mesmo com uma complexidade muito alta,
podemos resolver problemas de tamanho expressívos utilizando poucos recursos,
em pouco tempo. 

Conseguimos verificar propriedades de circuitos combinatórios digitais
simples. Utilizando um compilador de lógica proposicional para CNF poderiamos
verificar propriedades mais avançadas, como a releção entre propriedade
de input e output com o uso de implicações, com uma possível continuação
do desenvolvimento deste software. 

